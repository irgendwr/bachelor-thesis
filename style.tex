% Kodierung, Sprache, Patches
\usepackage[T1]{fontenc}    % Ausgabekodierung; ermöglicht Akzente und Umlaute
                            %  sowie korrekte Silbentrennung.
\usepackage[utf8]{inputenc} % Erlaubt die direkte Eingabe spezieller Zeichen;
                            %  utf8 muss die Eingabekodierung des Editors sein.
\usepackage[english]{babel} % Englische Sprachanpassungen (z.B. Überschriften).
\usepackage{microtype}      % Optimale Randausrichtung und Skalierung.
\usepackage[
    autostyle,
]{csquotes}                 % Korrekte Anführungszeichen in der Literaturliste.
\usepackage{scrhack}        % Verhindert Warnungen mit älteren Paketen.
\usepackage[
    newcommands
]{ragged2e}                 % Verbesserte \ragged...Befehle
\PassOptionsToPackage{
    hyphens
}{url}                      % Sorgt für URL-Umbrüche in Fußzeilen u. Literatur


% Schriftarten
\usepackage{mathptmx}       % Times; modifies the default serif and math fonts
\usepackage[scaled=.92]{helvet}% modifies the sans serif font
\usepackage{courier}        % modifies the monospace font


% Biblatex
\usepackage[
    style=alphabetic,
    backend=biber,
    %backref=true
]{biblatex}             % Biblatex mit alphabetischem Style und biber.
%\bibliography{\jobname.bib} % Dateiname der bib-Datei.
\DeclareFieldFormat*{title}{
    \mkbibemph{#1}          % Make titles italics
}


% Dokument- und Texteinstellungen
\usepackage[
    a4paper,
    margin=2.54cm,
    marginparwidth=2.0cm,
    footskip=1.0cm
]{geometry}                 % Ersetzt 'a4wide'.
\clubpenalty=10000          % Keine Einzelzeile am Beginn eines Absatzes
                            %  (Schusterjungen).
\widowpenalty=10000         % Keine Einzelzeile am Ende eines Absatzes
\displaywidowpenalty=10000  %  (Hurenkinder).
\usepackage{floatrow}       % Zentriert alle Floats
\usepackage{ifdraft}        % Ermöglicht \ifoptionfinal{true}{false}
\pagestyle{plain}           % keine Kopfzeilen
% \sloppy                   % großzügige Formatierungsweise
\deffootnote{1em}{1em}{
  \thefootnotemark.\ }      % Verbessert Layout mehrzeiliger Fußnoten

\makeatletter
\AtBeginDocument{
    \hypersetup{
        pdftitle = {\@title},
        pdfauthor = \@author,
    }
}
\makeatother


% Weitere Pakete
\usepackage{graphicx}       % Einfügen von Graphiken.
\usepackage{tabu}           % Einfügen von Tabellen.
\usepackage{multirow}       % Tabellenzeilen zusammenfassen.
\usepackage{multicol}       % Tabellenspalten zusammenfassen.
\usepackage{booktabs}       % Schönere Tabellen (\toprule\midrule\bottomrule).
\usepackage[nocut]{thmbox}  % Theorembox bspw. für Angreifermodell.
\usepackage{amsmath}        % Erweiterte Handhabung mathematischer Formeln.
\usepackage{amssymb}        % Erweiterte mathematische Symbole.
\usepackage{rotating}
\usepackage[
    printonlyused
]{acronym}                  % Abkürzungsverzeichnis
\usepackage[
    colorinlistoftodos,
    textsize=tiny,          % Notizen und TODOs - mit der todonotes.sty von
    \ifoptionfinal{disable}{}%  Benjamin Kellermann ist das Package "changebar"
]{todonotes}                %  bereits integriert.
\usepackage[
    breaklinks,
    hidelinks,
    pdfdisplaydoctitle,
    pdfpagemode = {UseOutlines},
    pdfpagelabels,
]{hyperref}                 % Sprungmarken im PDF. Lädt das URL-Paket.
\urlstyle{rm}               % Entfernt die Formattierung von URLs.

% Direktes Einfügen von Dateiinhalt. Wird hier für Verwendung
% einer .bib-Datei in dieser .tex-Datei benötigt.
%\usepackage{filecontents}

%\usepackage{breakurl}
%\def\UrlBreaks{\do\/\do-}

\usepackage{listings}       % Spezielle Umgebung für Quelltextformatierung.
\lstset{
    language=C,
    breaklines=true,
    breakatwhitespace=true,
    frame=l,            % Linie links: l, doppelt: L
    framerule=2.5pt,    % Dicke der Linie
    rulecolor=\color{gray},% Farbe der Linie
    captionpos=b,
    xleftmargin=6ex,
    tabsize=4,
    numbers=left,
    numberstyle=\ttfamily\footnotesize,
    basicstyle=\ttfamily\footnotesize,
    keywordstyle=\bfseries\color{green!50!black},
    commentstyle=\itshape\color{magenta!90!black},
    identifierstyle=\ttfamily,
    stringstyle=\color{orange!90!black},
    showstringspaces=false,
}

% CUSTOM
\usepackage{enumitem} % Used for description layout.
\usepackage{wrapfig} % Allows wrapping text around figures and tables.
\usepackage{tabularx}
\usepackage[page,toc,title]{appendix} % Appendices
\usepackage{pdflscape} % Landscape
\usepackage{siunitx} % Formats numbers and (SI) units: https://ctan.org/pkg/siunitx
\sisetup{
	round-mode = places,
	round-precision = 2
}

% Normal spacing between sentences.
\frenchspacing

% Capitalize chapter/section/subsection references.
\addto\extrasenglish{
  \def\chapterautorefname{Chapter}
  \def\sectionautorefname{Section}
  \def\subsectionautorefname{Subsection}
}

\renewcommand{\lstlistlistingname}{List of Listings}

\def\minline{\lstinline[language={},keywordstyle={},commentstyle={},identifierstyle={},stringstyle={}]}

\lstdefinelanguage{JavaScript}{
    keywords={typeof, new, true, false, try, function, return, null, catch, switch, var, const, let, if, in, while, do, else, case, break},
    ndkeywords={class, export, boolean, throw, implements, import, this},
    sensitive=false,
    comment=[l]{//},
    morecomment=[s]{/*}{*/},
    morestring=[b]',
    morestring=[b]",
    morestring=[b]`,
    numberstyle=\ttfamily\footnotesize,
    basicstyle=\ttfamily\footnotesize,
    keywordstyle=\bfseries\color{green!50!black},
    ndkeywordstyle=\color{darkgray}\bfseries,
    commentstyle=\itshape\color{magenta!90!black},
    identifierstyle=\ttfamily,
    stringstyle=\color{orange!90!black},
    showstringspaces=false,
}

\lstdefinelanguage{CSP}{
    alsodigit={-},
    keywords={child-src, connect-src, default-src, font-src, frame-src, img-src, manifest-src, media-src, object-src, prefetch-src, script-src, script-src-elem, script-src-attr, style-src, style-src-elem, style-src-attr, worker-src, base-uri, sandbox, form-action, frame-ancestors, navigate-to, report-uri, report-to, require-sri-for, require-trusted-types-for, trusted-types, upgrade-insecure-requests, block-all-mixed-content, plugin-types, referrer },
    ndkeywords={none, self, strict-dynamic, report-sample, unsafe-inline, unsafe-eval, unsafe-hashes, unsafe-allow-redirects, sha256-, sha384-, sha512-, nonce- },
    otherkeywords={;},
    sensitive=false,
    numberstyle=\ttfamily\footnotesize,
    basicstyle=\ttfamily\footnotesize,
    keywordstyle=\bfseries\color{green!50!black},
    ndkeywordstyle=\color{darkgray}\bfseries,
    commentstyle=\itshape\color{magenta!90!black},
    identifierstyle=\ttfamily,
    stringstyle=\color{orange!90!black},
    showstringspaces=false,
}

% alias
\newcommand{\icode}{\minline}
\newcommand{\browserAPI}{\hyperref[sec.browserAPIs]{browser API}}
\newcommand{\browserAPIs}{\hyperref[sec.browserAPIs]{browser APIs}}
\newcommand{\BrowserAPI}{\hyperref[sec.browserAPIs]{Browser API}}
\newcommand{\BrowserAPIs}{\hyperref[sec.browserAPIs]{Browser APIs}}
