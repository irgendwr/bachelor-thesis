\chapter{Conclusion and Outlook}
\label{sec.conclusion}

This thesis set out to assess the threats that browser \ac{api} overwriting poses and investigate the prevalence of browser \ac{api} overwriting on real-world websites. In order to determine the prevalence, a browser extension and an automated analysis tool were developed and used to conduct an empirical evaluation of the \num[round-precision=0]{16000} most popular websites of the Tranco list.

While \ac{api} overwriting requires an attacker to be able to execute JavaScript code in a web application, the related work and threat model presented here show that there are circumstances in which this is possible. The assessment of the potential threats posed by browser \ac{api} overwriting revealed that such overwriting can be used by an attacker to compromise the integrity and confidentiality of data from users of the affected web application. Depending on the web application, it is also possible to compromise the security of passwords and cryptographic keys generated by web applications.

The evaluation determined that it is common practice to overwrite \acsp{api}, even when an up-to-date browser is used. In total, \SI[round-precision=0]{61.60}{\percent} of the successfully processed domains modified at least one browser \ac{api}. The results also show that more than half (\SI[round-precision=0]{53.45}{\percent}) of the sites include external scripts that modify browser APIs. The most common use for \ac{api} overwriting is the tracking of user interactions for analytics purposes.

The thesis presented possible approaches to mitigate the threats of \ac{api} manipulation and made recommendations for web developers, administrators and users that reduce the attack surface in order to prevent malicious \ac{api} manipulation. The recommendations focus on preventing the execution of code from unknown sources and reducing the amount of external dependencies.

The browser extension developed as part of this thesis can be used by web developers to verify the behavior of their web applications. In addition, the automated analysis tool can be modified in order to continually analyze specific sites of a web application to ensure that it overwrites \acsp{api} only when intended and only from sources that are trusted.

Further research is needed to create automated systems that are able to reliably classify whether detected \ac{api} modifications are malicious or legitimate. It might be possible to apply similar approaches as used by antivirus software to generate and compare signatures against known malicious JavaScript code. As this might be defeated by code obfuscation, it would also be possible to make use of function hooking to trace the behavior of unknown code and classify it using heuristic algorithms.
