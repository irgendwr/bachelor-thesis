{\let\clearpage\relax
\chapter*{Abstract}
}

Modern web browsers offer a plethora of JavaScript \acsp{api}, which provide functionality ranging from sending \acs{http} requests to signing and encrypting data via the Web Cryptography \acs{api}. Web applications are highly dependent on these native browser \acsp{api} due to their ease of use, added functionality, and efficiency. The \acsp{api} are accessible through JavaScript as normal objects and functions and can be overwritten by all scripts executed within the same site. While polyfill libraries make legitimate use of this property, third-party code included in web applications is also able to overwrite the functions, which can allow attackers to overwrite APIs with malicious code and thus manipulate the behavior of the web application and grant access to its data.

This thesis assesses threats posed by browser \acs{api} overwriting and investigates its prevalance on real-world websites. In order to determine the prevalance of \acs{api} overwriting, a browser extension and an automated analysis tool were developed and used to conduct an empirical evaluation of the \num[round-precision=0]{16000} most popular websites of the Tranco list. The threats presented in this thesis show that \acs{api} overwriting allows attackers to gain access to private data, manipulate user interactions and cause a denial of service. The evaluation determined that it is common practice to overwrite \acsp{api}, with the most common usage being the tracking of user behavior for analytics purposes. As part of a case study, this thesis also reverse engineered the code responsible for a seemingly suspicious overwrite of a Cryptography \acs{api}.
